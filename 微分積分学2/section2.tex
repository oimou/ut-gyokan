\documentclass[./index]{subfiles}
\begin{document}
%==========
% 2.8
% ==========
\setcounter{section}{1}
\section{多変数関数の微分}
\setcounter{subsection}{7}
\subsection{テイラーの定理(2変数 version)}

\subsubsection{連鎖律と数学的帰納法}
\begin{screen}
    \begin{proposition}
        自然数$k=1,2,\dots,n+1$に対して
        \begin{equation}
            \label{proposition:1}
            \varphi^{(k)}(t)
            =
            \sum_{i=0}^{k}
                \combination{k}{i}
                \frac{\partial^k f}{\partial x^{k-i} \partial y^{i}} (x(t), y(t))
                \cdot
                (x_1 - x_0)^{k-i}
                (y_1 - y_0)^{i}
        \end{equation}
        が成立する.ただし$n, \varphi, f, x, y$は講義資料で定義されたものである.
    \end{proposition}
\end{screen}

この命題は,テイラーの定理(2変数 version)の証明のなかで
「同様に連鎖律をくり返し使うことにより…」と議論が省略されている部分です.

\begin{proof}
    $\Delta x = x_1 - x_0, \Delta y = y_1 - y_0$とおく.
    まず$k=1$に対して\cref{proposition:1}は明らかに成立する.
    つぎに$k\,(=1,2,\dots,n)$を任意にとり,\cref{proposition:1}の成立を仮定する.
    \cref{proposition:1}の両辺を$t$で微分して次を得る.
    \begin{alignat}{1}
        \varphi^{k+1}(t) &= 
            \sum_{i=0}^{k}
                \combination{k}{i}
                \bigg\{
                \frac{\partial^{k+1} f}{\partial x^{k+1-i} \partial y^{i}} (x(t), y(t))
                \Delta x^{k+1-i}
                \Delta y^{i} \nonumber \\
        &\quad +
                \frac{\partial^{k+1} f}{\partial x^{k-i} \partial y^{i+1}} (x(t), y(t))
                \Delta x^{k-i}
                \Delta y^{i+1}
                \bigg\} \\
        &= 
            \sum_{i=0}^{k}
                \combination{k}{i}
                \frac{\partial^{k+1} f}{\partial x^{k+1-i} \partial y^{i}} (x(t), y(t))
                \Delta x^{k+1-i}
                \Delta y^{i} \nonumber \\
        &\quad +
            \sum_{i=0}^{k}
                \combination{k}{i}
                \frac{\partial^{k+1} f}{\partial x^{k-i} \partial y^{i+1}} (x(t), y(t))
                \Delta x^{k-i}
                \Delta y^{i+1} \\
        % 一段ずらす
        &= 
            \sum_{i=0}^{k}
                \combination{k}{i}
                \frac{\partial^{k+1} f}{\partial x^{k+1-i} \partial y^{i}} (x(t), y(t))
                \Delta x^{k+1-i}
                \Delta y^{i} \nonumber \\
        &\quad +
            \sum_{i=1}^{k+1}
                \combination{k}{i-1}
                \frac{\partial^{k+1} f}{\partial x^{k+1-i} \partial y^{i}} (x(t), y(t))
                \Delta x^{k+1-i}
                \Delta y^{i} \\
        &= 
            \frac{\partial^{k+1} f}{\partial x^{k+1}} (x(t), y(t))
            \Delta x^{k+1}
            +
            \frac{\partial^{k+1} f}{\partial y^{k+1}} (x(t), y(t))
            \Delta y^{k+1} \nonumber \\
        &\quad +
            \sum_{i=1}^{k}
                \combination{k+1}{i}
                \frac{\partial^{k+1} f}{\partial x^{k+1-i} \partial y^{i}} (x(t), y(t))
                \Delta x^{k+1-i}
                \Delta y^{i} \\
        &=
            \sum_{i=0}^{k+1}
                \combination{k+1}{i}
                \frac{\partial^{k+1} f}{\partial x^{k+1-i} \partial y^{i}} (x(t), y(t))
                \Delta x^{k+1-i}
                \Delta y^{i}
    \end{alignat}
    したがって$k+1$に対しても\cref{proposition:1}が成立する.
    以上より,自然数$k=1,2,\dots,n+1$に対して\cref{proposition:1}が成立する.
\end{proof}

% ==========
% 2.9
% ==========
\subsection{$C^2$級の2変数関数の極大・極小}

\subsubsection{極値を取るための必要条件}
\begin{screen}
    \begin{proposition}
        点$(x_0, y_0)$で$f$が極大,または極小になるならば
        \begin{equation}
            \frac{\partial f}{\partial x}(x_0, y_0) = 0
            \quad \mbox{かつ} \quad
            \frac{\partial f}{\partial y}(x_0, y_0) = 0
        \end{equation}
        である.ただし$x_0, y_0, f$は講義資料で定義されたものである.
    \end{proposition}
\end{screen}

この命題は「すぐにわかる」ものです.

\begin{proof}
    点$(x_0, y_0)$が$f$の極大点である場合を示す.極小点の場合も同様である.
    一変数関数$g, h$を$g(x) = f(x, y_0),\,h(y) = f(x_0, y)$と定めると,
    $g, h$はそれぞれ点$x_0, y_0$で微分可能であって局所的に最大となる.
    したがって,共通資料第4章命題2より
    点$x_0, y_0$はそれぞれ$g, h$の停留点である.よって
    \begin{equation}
        \frac{\partial f}{\partial x}(x_0, y_0) = g'(x_0) = 0
        ,\quad
        \frac{\partial f}{\partial y}(x_0, y_0) = h'(y_0) = 0
    \end{equation}
    が従う.
\end{proof}

\subsubsection{$A, B, C$ と $a, b, c$ の関係性}
\begin{screen}
    \begin{proposition}
        $(x_0, y_0)$に十分近い任意の点$(x, y)$に対して
        \begin{equation}
            A > 0
            \,\mbox{かつ}\,
            AC - B^2 > 0
            \quad \Longrightarrow \quad
            \mbox{常に}\,
            a > 0
            \,\mbox{かつ}\,
            ac - b^2 > 0
        \end{equation}
        が成立する.ただし$x, y, x_0, y_0, A, B, C, a, b, c$は講義資料で定義されたものである.
    \end{proposition}
\end{screen}

この命題は,$A, B, C$に関する大小関係がまわりの$a, b, c$に関しても成立するというものです.

\begin{proof}
    $(x, y) \rightarrow (x_0, y_0)$で
    \begin{alignat}{1}
        a &\rightarrow A > 0 \\
        ac - b^2 &\rightarrow AC - B^2 > 0
    \end{alignat}
    なので,ある$\delta > 0$が存在して
    \begin{equation}
        0 < \| (x, y) - (x_0, y_0) \| < \delta
        \quad \Longrightarrow \quad
        a > 0
        \,\mbox{かつ}\,
        ac - b^2 > 0
    \end{equation}
    が成立する.
\end{proof}

\subsubsection{「$|t|$: 十分小」の意味}
\begin{screen}
    \begin{proposition}
        「$|t|$: 十分小」という制約は,$f$が$C^2$級である領域の上だけを点$(x_0 + t, y_0)$が動くように課されている.
    \end{proposition}
\end{screen}

「$AC-B^2<0$の場合」というスライドでは「$|t|$: 十分小」という制約が登場しますが,
これは何のためにあるのか気になりませんか?
O先生に尋ねたところ上のような回答でした.証明はとくにありません.

\subsubsection{「iii) $A=0$のとき」の省略された計算}
\begin{screen}
    \begin{proposition}
        講義資料のように
        \begin{equation}
            \varphi(t) = f(x_0 + p_1 t, y_0 + t),
            \quad
            \psi(t)    = f(x_0 + p_2 t, y_0 + t)
        \end{equation}
        とおくと次が成り立つ.
        \begin{equation}
            \varphi'(0) = \psi'(0) = 0,
            \quad
            \varphi''(0) = 2p_1 B + C,
            \quad
            \psi''(0)    = 2p_2 B + C
        \end{equation}
        ただし$x_0, y_0, p_1, p_2, B, C$は講義資料で定義されたものである.
    \end{proposition}
\end{screen}

\begin{proof}
    $\varphi$ についてのみ示す.$\psi$ の場合も同様である.
    $f$が$C^2$級関数であることに注意すれば,$\varphi$の導関数は
    \begin{alignat}{1}
        \varphi'(t)
        &=
              \frac{\partial f}{\partial x}(x_0 + p_1 t, y_0 + t) \cdot p_1
            + \frac{\partial f}{\partial y}(x_0 + p_1 t, y_0 + t) \cdot 1 \label{proposition:5:1} \\
        \varphi''(t)
        &=
              \frac{\partial^2 f}{\partial x^2}(x_0 + p_1 t, y_0 + t) \cdot p_1^2 \nonumber \\
        &\quad +
            2 \frac{\partial^2 f}{\partial x \partial y}(x_0 + p_1 t, y_0 + t) \cdot p_1
            + \frac{\partial^2 f}{\partial y^2}(x_0 + p_1 t, y_0 + t) \cdot 1 \label{proposition:5:2}
    \end{alignat}
    である.点$(x_0, y_0)$は$f$の停留点なので,\cref{proposition:5:1}より$\varphi'(0) = 0$である.
    また,$A, B, C$の定義と$A = 0$に注意すれば,\cref{proposition:5:2}より
    \begin{equation}
        \varphi''(0) = 0 \cdot p_1^2 + 2B \cdot p_1 + C \cdot 1 = 2p_1 B + C
    \end{equation}
    を得る.
\end{proof}
\end{document}