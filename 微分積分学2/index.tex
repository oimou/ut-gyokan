\documentclass[mingoth]{jsarticle}
\usepackage{parskip}
\usepackage{ascmac}
\usepackage{amsmath,amssymb}
\usepackage{cleveref}
    \crefname{equation}{}{}
    \crefname{proposition}{行間}{行間}
    \crefname{theorem}{定理}{定理}
    \crefname{lemma}{補題}{補題}
    \crefmultiformat{proposition}{行間~#2#1#3}{,~#2#1#3}{, #2#1#3}{,~#2#1#3}
\usepackage{amsthm}
    \makeatletter
    \renewenvironment{proof}[1][\proofname]{\par
        \pushQED{\qed}
        \normalfont
        \topsep6\p@\@plus6\p@ \trivlist
        \item[\hskip\labelsep{\bfseries #1}\@addpunct{\bfseries}]\ignorespaces
    }{%
        \popQED\endtrivlist\@endpefalse
    }
    \renewcommand{\proofname}{\underline{証明.}}
    \renewcommand{\qedsymbol}{$\blacksquare$}
    \makeatother
\usepackage{bm}
\usepackage{afterpage}
\usepackage{url}
\usepackage{enumerate}
\usepackage[shortlabels]{enumitem}
\usepackage{fancyhdr}
    \pagestyle{fancy}
    \lhead{微分積分学② 行間埋め}
    \rhead{yahata}

\newtheorem{proposition}{行間}
\newtheorem{theorem}{定理}
\newtheorem{lemma}{補題}
\newcommand{\combination}[2]{{}_{#1} \mathrm{C}_{#2}}

\begin{document}

% ==========
% Description
% ==========
微分積分学②(火曜4限; O先生)の講義資料の行間を埋める資料です.
\tableofcontents
\newpage

% ==========
% 2.8
% ==========
\setcounter{section}{1}
\section{多変数関数の微分}
\setcounter{subsection}{7}
\subsection{テイラーの定理(2変数 version)}

\subsubsection{連鎖律と数学的帰納法}
\begin{screen}
    \begin{proposition}
        自然数$k=1,2,\dots,n+1$に対して
        \begin{equation}
            \label{proposition:1}
            \varphi^{(k)}(t)
            =
            \sum_{i=0}^{k}
                \combination{k}{i}
                \frac{\partial^k f}{\partial x^{k-i} \partial y^{i}} (x(t), y(t))
                \cdot
                (x_1 - x_0)^{k-i}
                (y_1 - y_0)^{i}
        \end{equation}
        が成立する.ただし$n, \varphi, f, x, y$は講義資料で定義されたものである.
    \end{proposition}
\end{screen}

この命題は,テイラーの定理(2変数 version)の証明のなかで
「同様に連鎖律をくり返し使うことにより…」と議論が省略されている部分です.

\begin{proof}
    $\Delta x = x_1 - x_0, \Delta y = y_1 - y_0$とおく.
    まず$k=1$に対して\cref{proposition:1}は明らかに成立する.
    つぎに$k\,(=1,2,\dots,n)$を任意にとり,\cref{proposition:1}の成立を仮定する.
    \cref{proposition:1}の両辺を$t$で微分して次を得る.
    \begin{alignat}{1}
        \varphi^{k+1}(t) &= 
            \sum_{i=0}^{k}
                \combination{k}{i}
                \bigg\{
                \frac{\partial^{k+1} f}{\partial x^{k+1-i} \partial y^{i}} (x(t), y(t))
                \Delta x^{k+1-i}
                \Delta y^{i} \nonumber \\
        &\quad +
                \frac{\partial^{k+1} f}{\partial x^{k-i} \partial y^{i+1}} (x(t), y(t))
                \Delta x^{k-i}
                \Delta y^{i+1}
                \bigg\} \\
        &= 
            \sum_{i=0}^{k}
                \combination{k}{i}
                \frac{\partial^{k+1} f}{\partial x^{k+1-i} \partial y^{i}} (x(t), y(t))
                \Delta x^{k+1-i}
                \Delta y^{i} \nonumber \\
        &\quad +
            \sum_{i=0}^{k}
                \combination{k}{i}
                \frac{\partial^{k+1} f}{\partial x^{k-i} \partial y^{i+1}} (x(t), y(t))
                \Delta x^{k-i}
                \Delta y^{i+1} \\
        % 一段ずらす
        &= 
            \sum_{i=0}^{k}
                \combination{k}{i}
                \frac{\partial^{k+1} f}{\partial x^{k+1-i} \partial y^{i}} (x(t), y(t))
                \Delta x^{k+1-i}
                \Delta y^{i} \nonumber \\
        &\quad +
            \sum_{i=1}^{k+1}
                \combination{k}{i-1}
                \frac{\partial^{k+1} f}{\partial x^{k+1-i} \partial y^{i}} (x(t), y(t))
                \Delta x^{k+1-i}
                \Delta y^{i} \\
        &= 
            \frac{\partial^{k+1} f}{\partial x^{k+1}} (x(t), y(t))
            \Delta x^{k+1}
            +
            \frac{\partial^{k+1} f}{\partial y^{k+1}} (x(t), y(t))
            \Delta y^{k+1} \nonumber \\
        &\quad +
            \sum_{i=1}^{k}
                \combination{k+1}{i}
                \frac{\partial^{k+1} f}{\partial x^{k+1-i} \partial y^{i}} (x(t), y(t))
                \Delta x^{k+1-i}
                \Delta y^{i} \\
        &=
            \sum_{i=0}^{k+1}
                \combination{k+1}{i}
                \frac{\partial^{k+1} f}{\partial x^{k+1-i} \partial y^{i}} (x(t), y(t))
                \Delta x^{k+1-i}
                \Delta y^{i}
    \end{alignat}
    したがって$k+1$に対しても\cref{proposition:1}が成立する.
    以上より,自然数$k=1,2,\dots,n+1$に対して\cref{proposition:1}が成立する.
\end{proof}

% ==========
% 2.9
% ==========
\subsection{$C^2$級の2変数関数の極大・極小}

\subsubsection{極値を取るための必要条件}
\begin{screen}
    \begin{proposition}
        点$(x_0, y_0)$で$f$が極大,または極小になるならば
        \begin{equation}
            \frac{\partial f}{\partial x}(x_0, y_0) = 0
            \quad \mbox{かつ} \quad
            \frac{\partial f}{\partial y}(x_0, y_0) = 0
        \end{equation}
        である.ただし$x_0, y_0, f$は講義資料で定義されたものである.
    \end{proposition}
\end{screen}

この命題は「すぐにわかる」ものです.

\begin{proof}
    点$(x_0, y_0)$が$f$の極大点である場合を示す.極小点の場合も同様である.
    一変数関数$g, h$を$g(x) = f(x, y_0),\,h(y) = f(x_0, y)$と定めると,
    $g, h$はそれぞれ点$x_0, y_0$で微分可能であって局所的に最大となる.
    したがって,共通資料第4章命題2より
    点$x_0, y_0$はそれぞれ$g, h$の停留点である.よって
    \begin{equation}
        \frac{\partial f}{\partial x}(x_0, y_0) = g'(x_0) = 0
        ,\quad
        \frac{\partial f}{\partial y}(x_0, y_0) = h'(y_0) = 0
    \end{equation}
    が従う.
\end{proof}

\subsubsection{$A, B, C$ と $a, b, c$ の関係性}
\begin{screen}
    \begin{proposition}
        $(x_0, y_0)$に十分近い任意の点$(x, y)$に対して
        \begin{equation}
            A > 0
            \,\mbox{かつ}\,
            AC - B^2 > 0
            \quad \Longrightarrow \quad
            \mbox{常に}\,
            a > 0
            \,\mbox{かつ}\,
            ac - b^2 > 0
        \end{equation}
        が成立する.ただし$x, y, x_0, y_0, A, B, C, a, b, c$は講義資料で定義されたものである.
    \end{proposition}
\end{screen}

この命題は,$A, B, C$に関する大小関係がまわりの$a, b, c$に関しても成立するというものです.

\begin{proof}
    $(x, y) \rightarrow (x_0, y_0)$で
    \begin{alignat}{1}
        a &\rightarrow A > 0 \\
        ac - b^2 &\rightarrow AC - B^2 > 0
    \end{alignat}
    なので,ある$\delta > 0$が存在して
    \begin{equation}
        0 < \| (x, y) - (x_0, y_0) \| < \delta
        \quad \Longrightarrow \quad
        a > 0
        \,\mbox{かつ}\,
        ac - b^2 > 0
    \end{equation}
    が成立する.
\end{proof}

\subsubsection{「$|t|$: 十分小」の意味}
\begin{screen}
    \begin{proposition}
        「$|t|$: 十分小」という制約は,$f$が$C^2$級である領域の上だけを点$(x_0 + t, y_0)$が動くように課されている.
    \end{proposition}
\end{screen}

「$AC-B^2<0$の場合」というスライドでは「$|t|$: 十分小」という制約が登場しますが,
これは何のためにあるのか気になりませんか?
O先生に尋ねたところ上のような回答でした.証明はとくにありません.

\subsubsection{「iii) $A=0$のとき」の省略された計算}
\begin{screen}
    \begin{proposition}
        講義資料のように
        \begin{equation}
            \varphi(t) = f(x_0 + p_1 t, y_0 + t),
            \quad
            \psi(t)    = f(x_0 + p_2 t, y_0 + t)
        \end{equation}
        とおくと次が成り立つ.
        \begin{equation}
            \varphi'(0) = \psi'(0) = 0,
            \quad
            \varphi''(0) = 2p_1 B + C,
            \quad
            \psi''(0)    = 2p_2 B + C
        \end{equation}
        ただし$x_0, y_0, p_1, p_2, B, C$は講義資料で定義されたものである.
    \end{proposition}
\end{screen}

\begin{proof}
    $\varphi$ についてのみ示す.$\psi$ の場合も同様である.
    $f$が$C^2$級関数であることに注意すれば,$\varphi$の導関数は
    \begin{alignat}{1}
        \varphi'(t)
        &=
              \frac{\partial f}{\partial x}(x_0 + p_1 t, y_0 + t) \cdot p_1
            + \frac{\partial f}{\partial y}(x_0 + p_1 t, y_0 + t) \cdot 1 \label{proposition:5:1} \\
        \varphi''(t)
        &=
              \frac{\partial^2 f}{\partial x^2}(x_0 + p_1 t, y_0 + t) \cdot p_1^2 \nonumber \\
        &\quad +
            2 \frac{\partial^2 f}{\partial x \partial y}(x_0 + p_1 t, y_0 + t) \cdot p_1
            + \frac{\partial^2 f}{\partial y^2}(x_0 + p_1 t, y_0 + t) \cdot 1 \label{proposition:5:2}
    \end{alignat}
    である.点$(x_0, y_0)$は$f$の停留点なので,\cref{proposition:5:1}より$\varphi'(0) = 0$である.
    また,$A, B, C$の定義と$A = 0$に注意すれば,\cref{proposition:5:2}より
    \begin{equation}
        \varphi''(0) = 0 \cdot p_1^2 + 2B \cdot p_1 + C \cdot 1 = 2p_1 B + C
    \end{equation}
    を得る.
\end{proof}

% ==========
% 3.1
% ==========
\section{1変数関数の積分}
\setcounter{subsection}{1}
\subsection{区分求積法}
\subsubsection{区分求積法の成立}
\begin{screen}
    \begin{proposition}
        関数$f:[a,b] \rightarrow \mathbb{R}$がリーマン積分可能であるとき,
        $\lim_{n\rightarrow\infty} |\triangle_n| = 0$なる
        $[a, b]$の分割の列$\{\triangle_n\}_{n=1,2,\dots}$を任意にとり,
        各$\triangle_n$の代表点集合$\mathbf{\xi_n}$を任意にとれば,
        \begin{equation}
            \lim_{n\rightarrow\infty} R(f: \triangle_n, \mathbf{\xi_n})
            =
            \int_a^b f(x) dx
        \end{equation}
        が成立する.
    \end{proposition}
\end{screen}

こちらはリーマン積分可能性の定義から区分求積法の成立を導くものです.

\begin{proof}
    $S = \int_a^b f(x) dx$とおく.
    正数$\epsilon$を任意にとる.
    $f$はリーマン積分可能なので,ある正数$\delta$であって次を満たすものが存在する.
    \begin{alignat}{1}
        \begin{cases}
            |\triangle| < \delta \mbox{ なる } [a, b] \mbox{ の任意の分割 } \triangle \\
            \triangle \mbox{ の任意の代表点集合 } \mathbf{\xi}
        \end{cases}
        \mbox{に対し} \quad
        |R(f: \triangle, \mathbf{\xi}) - S| < \epsilon
        \label{proposition:6:1}
    \end{alignat}
    このような$\delta$をひとつとる.
    $\lim_{n\rightarrow\infty} |\triangle_n| = 0$より,ある自然数$N$であって次を満たすものが存在する.
    \begin{equation}
        n > N \Longrightarrow |\triangle_n| < \delta
        \label{proposition:6:2}
    \end{equation}
    このような$N$をひとつとる.
    自然数$n > N$を任意にとる.
    \cref{proposition:6:2},\cref{proposition:6:1}より
    \begin{equation}
        |R(f: \triangle_n, \mathbf{\xi_n}) - S| < \epsilon
    \end{equation}
    が成立する.すなわち
    \begin{equation}
        \lim_{n\rightarrow\infty} R(f: \triangle_n, \mathbf{\xi_n})
        =
        S
        =
        \int_a^b f(x) dx
    \end{equation}
    が成立する.
\end{proof}

% ==========
% 3.4
% ==========
\setcounter{subsection}{3}
\subsection{有界な関数のリーマン積分可能性・不可能性}
\subsubsection{すぐに分かること2}
\begin{screen}
    \begin{proposition}
        区間$[a, b]$の2つの分割$\triangle_1$と$\triangle_2$について,
        $\triangle_2$が$\triangle_1$の細分ならば
        \begin{equation}
            s(f:\triangle_1) \leq s(f:\triangle_2) \leq S(f: \triangle_2) \leq S(f: \triangle_1)
            \label{proposition:7}
        \end{equation}
        が成立する.
    \end{proposition}
\end{screen}

この命題は「簡単なので略」されています.

\begin{proof}
    \cref{proposition:7}の最も左の不等号についてのみ示す.
    分割$\triangle_1$の隣り合う分点$x_{j-1}$と$x_{j}$の間に分点$x'$を追加することを考える.
    区間$[x_{j-1}, x_{j}], [x_{j-1}, x'], [x', x_j]$上での$f$の下限をそれぞれ
    $m_j, m_j', m_{j+1}'$とおく.ここで,
    \begin{equation}
        m_j',\, m_{j+1}' \geq m_j
    \end{equation}
    ゆえに
    \begin{alignat}{1}
        s(f:\triangle_2) - s(f:\triangle_1) &=
            m_j' (x' - x_{j-1}) + m_{j+1}' (x_j - x') - m_j (x_j - x_{j-1}) \\
        &\geq
            m_j (x' - x_{j-1}) + m_j (x_j - x') - m_j (x_j - x_{j-1}) \\
        &=
            0
    \end{alignat}
    が成立する.
\end{proof}

\subsubsection{有界閉区間上の連続関数の一様連続性}
\begin{screen}
    \begin{proposition}
        関数$f:D\rightarrow\mathbb{R}$について,
        $D$が$\mathbb{R}$の有界閉集合かつ
        $f$が連続であるならば,
        $f$は一様連続である.
    \end{proposition}
\end{screen}

これは証明が解析学基礎に投げられている定理です.

\begin{proof}
    数学IA演習(2014年度)第9回講義資料
    \footnote{\url{https://lecture.ecc.u-tokyo.ac.jp/~nkiyono/14_kami.html}}
    \footnote{K先生の資料は神です}
    の定理8で証明されている.
\end{proof}

\subsubsection{リーマンの判定法}
\begin{screen}
    \begin{proposition}
        \label{proposition:9}
        有界関数$f:[a,b]\rightarrow\mathbb{R}$について, \\
        $\forall \epsilon > 0$に対し,区間$[a,b]$のある分割$\triangle$が存在して
        $S(f: \triangle) - s(f: \triangle) < \epsilon$となる. \\
        \quad $\Longleftrightarrow$ $f$が区間$[a,b]$上でリーマン積分可能
    \end{proposition}
\end{screen}

これは「ダルブーの定理」というものが必要です,として証明が省略されていますが,
リーマンの判定法の証明のためにはダルブーの定理以外にも色々と補題を準備しておかないといけません.

\begin{proof}
    数学IA演習(2014年度)第9回講義資料
    \footnote{\url{https://lecture.ecc.u-tokyo.ac.jp/~nkiyono/14_kami.html}}
    で証明されている.
    以下,講義資料内の定理番号を用いて証明の流れを示す.

    \cref{proposition:9}の主張は直接的には
    定理13「積分可能条件:$\epsilon$-$\delta$バージョン」として証明されるが,
    もともとのリーマン積分可能性の定義と定理13のいう積分可能条件との同値性は次のように示される.
    ただし,同値記号の下に書き添えてある定理は,その同値性を示すために用いられる定理である.
    \begin{align*}
        \mbox{定理13のいう積分可能条件}
            &\Longleftrightarrow \mbox{定理12のいう積分可能条件} \\
            &\underbrace{\Longleftrightarrow}_{\mbox{\scriptsize 定理11}} \mbox{定理6のいう積分可能条件} \\
            &\Longleftrightarrow \mbox{定理4のいう積分可能条件} \\
            &\underbrace{\Longleftrightarrow}_{\mbox{\scriptsize 定理2,3}} \mbox{定義}
    \end{align*}
\end{proof}

% ==========
% 3.5
% ==========
\subsection{リーマン積分の基本性質}
\subsubsection{リーマン積分の線型性(1)}

\begin{screen}
    \begin{proposition}
        関数$f:[a,b]\rightarrow\mathbb{R}$,$g:[a,b]\rightarrow\mathbb{R}$がともに
        リーマン積分可能であるとき,
        関数$f+g: [a,b] \rightarrow \mathbb{R} \, (x \mapsto f(x) + g(x))$もリーマン積分可能で
        \begin{equation}
            \int_a^b (f(x) + g(x)) dx = \int_a^b f(x) dx + \int_a^b g(x) dx
            \label{proposition:10:1}
        \end{equation}
    \end{proposition}
\end{screen}

\begin{proof}
    表記の簡略化のため$\alpha = \int_a^b f(x) dx$,$\beta = \int_a^b g(x) dx$とおく.

    正数$\epsilon$を任意にとる.
    $f, g$がリーマン積分可能であることから,ある正数$\delta$が存在して
    \begin{equation}
        \nonumber
        \begin{cases}
            |\triangle| < \delta \mbox{ なる } [a, b] \mbox{ の任意の分割 } \triangle \\
            \triangle \mbox{ の任意の代表点集合 } \mathbf{\xi}
        \end{cases}
        \mbox{に対し} \quad
        \begin{cases}
            |R(f: \triangle, \mathbf{\xi}) - \alpha| < \epsilon / 2 \\
            |R(f: \triangle, \mathbf{\xi}) - \beta | < \epsilon / 2
        \end{cases}
    \end{equation}
    が成立する.ここで,リーマン和の定義から明らかに
    \begin{equation}
        R(f + g: \triangle, \mathbf{\xi}) = R(f: \triangle, \mathbf{\xi}) + R(g: \triangle, \mathbf{\xi})
    \end{equation}
    なので
    \begin{alignat}{1}
        |R(f + g: \triangle, \mathbf{\xi}) - (\alpha + \beta)|
            &= |R(f: \triangle, \mathbf{\xi}) + R(g: \triangle, \mathbf{\xi}) - (\alpha + \beta)| \\
            &\leq |R(f: \triangle, \mathbf{\xi}) - \alpha| + |R(g: \triangle, \mathbf{\xi}) - \beta| \\
            &< \epsilon
    \end{alignat}
    が成立する.したがって\cref{proposition:10:1}が成立する.
\end{proof}

\subsubsection{リーマン積分の線型性(2)}

\begin{screen}
    \begin{proposition}
        関数$f:[a,b]\rightarrow\mathbb{R}$がリーマン積分可能であるとき,
        $\forall c \in \mathbb{R}$に対し,
        関数$cf: [a,b] \rightarrow \mathbb{R} \, (x \mapsto cf(x))$もリーマン積分可能で
        \begin{equation}
            \int_a^b cf(x) dx = c \int_a^b f(x) dx
            \label{proposition:10:2}
        \end{equation}
    \end{proposition}
\end{screen}

\begin{proof}
    表記の簡略化のため$\alpha = \int_a^b f(x) dx$とおく.

    $c \in \mathbb{R}$を任意にとる.
    正数$\epsilon$を任意にとる.
    $f$がリーマン積分可能であることから,ある正数$\delta$が存在して
    \begin{equation}
        \nonumber
        \begin{cases}
            |\triangle| < \delta \mbox{ なる } [a, b] \mbox{ の任意の分割 } \triangle \\
            \triangle \mbox{ の任意の代表点集合 } \mathbf{\xi}
        \end{cases}
        \mbox{に対し} \quad
        |R(f: \triangle, \mathbf{\xi}) - \alpha| < \dfrac{\epsilon}{|c| + 1}
    \end{equation}
    が成立する.ここで,リーマン和の定義から明らかに
    \begin{equation}
        R(cf: \triangle, \mathbf{\xi}) = cR(f: \triangle, \mathbf{\xi})
    \end{equation}
    なので
    \begin{alignat}{1}
        |R(cf: \triangle, \mathbf{\xi}) - c\alpha|
            &= |cR(f: \triangle, \mathbf{\xi}) - c\alpha| \\
            &= |c||R(f: \triangle, \mathbf{\xi}) - \alpha| \\
            &\leq |c| \dfrac{\epsilon}{|c| + 1} \\
            &< \epsilon
    \end{alignat}
    が成立する.したがって\cref{proposition:10:2}が成立する.
\end{proof}

\subsubsection{積のリーマン積分可能性}

\begin{screen}
    \begin{proposition}
        関数$f:[a,b] \rightarrow \mathbb{R}$,$g:[a,b] \rightarrow \mathbb{R}$が
        ともにリーマン積分可能であるとき,関数$fg:[a,b] \rightarrow \mathbb{R}\,(x \mapsto f(x)g(x))$も
        リーマン積分可能である.
    \end{proposition}
\end{screen}

この証明にはリーマンの判定法を利用します.

\begin{proof}
    まず
    \begin{equation}
        M = \max \left\{ \sup_{x \in [a,b]} |f(x)|,\, \sup_{x \in [a,b]} |g(x)| \right\}
    \end{equation}
    とおく\footnote{$f, g$の有界性により上限が存在します.}.
    $M = 0$の場合は$f = g = 0$ゆえに明らかに主張が成立するから,$M > 0$の場合を考える.

    正数$\epsilon$を任意にとる.
    $f, g$はリーマン積分可能なので,リーマンの判定法によれば
    $[a,b]$の分割$\triangle_f, \triangle_g$が存在して
    \begin{alignat}{1}
        S(f: \triangle_f) - s(f: \triangle_f) &< \frac{\epsilon}{2M}, \\
        S(g: \triangle_g) - s(g: \triangle_g) &< \frac{\epsilon}{2M}
    \end{alignat}
    が成立する.$\triangle_f, \triangle_g$の分点をあわせた分割を
    $\triangle: a = x_0 < x_1 < \dots < x_n = b$とおく.

    $\triangle$に関する関数$fg$の上限和と下限和の差を評価する.まず
    \begin{alignat}{1}
        S(fg: \triangle) - s(fg: \triangle)
            &= \sum_{i=1}^n \sup_{x \in [x_{i-1}, x_i]} f(x)g(x) (x_i - x_{i-1})
            - \sum_{i=1}^n \inf_{x \in [x_{i-1}, x_i]} f(x)g(x) (x_i - x_{i-1}) \\
        &= \sum_{i=1}^n \sup_{x, y \in [x_{i-1}, x_i]} |f(x)g(x) - f(y)g(y)| (x_i - x_{i-1})
        \label{proposition:12:1}
    \end{alignat}
    である(\cref{lemma:1}).ここで
    \begin{alignat}{1}
        |f(x)g(x) - f(y)g(y)| &= |g(x)\{f(x) - f(y)\} + f(y)\{g(x) - g(y)\}| \\
            &\leq |g(x)||f(x) - f(y)| + |f(y)||g(x) - g(y)| \\
            &\leq M|f(x) - f(y)| + M|g(x) - g(y)|
    \end{alignat}
    により
    \begin{alignat}{1}
        (\mbox{\cref{proposition:12:1}}式)
            &\leq \sum_{i=1}^n \sup_{x, y \in [x_{i-1}, x_i]} \{M|f(x) - f(y)| + M|g(x) - g(y)|\} (x_i - x_{i-1}) \\
            &= M \sum_{i=1}^n \sup_{x \in [x_{i-1}, x_i]} |f(x) - f(y)|(x_i - x_{i-1})
                + M \sum_{i=1}^n \sup_{x \in [x_{i-1}, x_i]} |g(x) - g(y)|(x_i - x_{i-1}) \\
            &= M \{S(f: \triangle) - s(f: \triangle)\} + M \{S(g: \triangle) - s(g: \triangle)\}
            \label{proposition:12:2}
    \end{alignat}
    が成立する.$\triangle$は$\triangle_f, \triangle_g$の細分であることに注意すれば
    \begin{alignat}{1}
        (\mbox{\cref{proposition:12:2}式})
            &\leq M \{S(f: \triangle_f) - s(f: \triangle_f)\} + M \{S(g: \triangle_g) - s(g: \triangle_g)\} \\
            &\leq M \frac{\epsilon}{2M} + M \frac{\epsilon}{2M} \\
            &= \epsilon
    \end{alignat}
    が成立する.
    すなわち
    \begin{equation}
        S(fg: \triangle) - s(fg: \triangle) < \epsilon
    \end{equation}
    である.したがって,リーマンの判定法により$fg$はリーマン積分可能である.
\end{proof}

\begin{screen}
    \begin{lemma}
        \label{lemma:1}
        有界な実関数$f$と$f$の定義域の任意の部分集合$A$に対して
        \begin{equation}
            \sup_{x \in A} f(x) - \inf_{x \in A} f(x) = \sup_{x,y \in A} |f(x) - f(y)|
        \end{equation}
        が成立する.
    \end{lemma}
\end{screen}

\begin{proof}
    \begin{alignat}{1}
        \sup_{x \in A} f(x) - \inf_{x \in A} f(x)
            &= \sup_{x \in A} f(x) + \sup_{x \in A} \{-f(x)\} \\
            &= \sup_{x,y \in A} \{f(x) - f(y)\} \\
            &= \sup_{x,y \in A} |f(x) - f(y)|
    \end{alignat}
\end{proof}

\end{document}
