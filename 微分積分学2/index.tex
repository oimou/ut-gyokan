\documentclass{article}
\usepackage{plext}
\usepackage{subfiles}
\usepackage[dvipdfmx]{graphicx}
\usepackage{parskip}
\usepackage{ascmac}
\usepackage{amsmath,amssymb}
\usepackage{cleveref}
    \crefname{proposition}{命題}{命題}
    \crefname{theorem}{定理}{定理}
    \crefname{lemma}{行間}{行間}
    \crefmultiformat{lemma}{行間~#2#1#3}{,~#2#1#3}{, #2#1#3}{,~#2#1#3}
\usepackage{amsthm}
    \makeatletter
    \renewenvironment{proof}[1][\proofname]{\par
        \pushQED{\qed}
        \normalfont
        \topsep6\p@\@plus6\p@ \trivlist
        \item[\hskip\labelsep{\bfseries #1}\@addpunct{\bfseries}]\ignorespaces
    }{%
        \popQED\endtrivlist\@endpefalse
    }
    \renewcommand{\proofname}{\underline{証明.}}
    \renewcommand{\qedsymbol}{$\blacksquare$}
    \makeatother
\usepackage{listings}
\usepackage{siunitx}
\usepackage{mhchem}
\usepackage{bm}
\usepackage{afterpage}
\usepackage{url}
\usepackage{fancyhdr}
    \pagestyle{fancy}
    \lhead{微分積分学② 行間埋め}
    \rhead{yahata}

\newtheorem{proposition}{命題}
\newtheorem{theorem}{定理}
\newtheorem{lemma}{行間}
\newcommand{\combination}[2]{{}_{#1} \mathrm{C}_{#2}}

\begin{document}

% ==========
% Description
% ==========
微分積分学②(火曜4限; O先生)の講義資料の行間を埋める資料です。
\tableofcontents
\newpage

% ==========
% Section 8
% ==========
\setcounter{section}{7}
\section{テイラーの定理(2変数 version)}

\subsection{連鎖律と数学的帰納法}
次の命題は、テイラーの定理(2変数 version)の証明のなかで
「同様に連鎖律をくり返し使うことにより…」と議論が省略されている部分です。
\begin{screen}
    \begin{lemma}
        自然数$k=1,2,\dots,n+1$に対して
        \begin{equation}
            \label{lemma:1}
            \varphi^{(k)}(t)
            =
            \sum_{i=0}^{k}
                \combination{k}{i}
                \frac{\partial^k f}{\partial x^{k-i} \partial y^{i}} (x(t), y(t))
                \cdot
                (x_1 - x_0)^{k-i}
                (y_1 - y_0)^{i}
        \end{equation}
        が成立する。ただし$n, \varphi, f, x, y$は講義資料で定義されたものである。
    \end{lemma}
\end{screen}

\begin{proof}
    $\Delta x = x_1 - x_0, \Delta y = y_1 - y_0$とおく。
    まず$k=1$に対して\cref{lemma:1}は明らかに成立する。
    つぎに$k\,(=1,2,\dots,n)$を任意にとり、\cref{lemma:1}の成立を仮定する。
    \cref{lemma:1}の両辺を$t$で微分して次を得る。
    \begin{alignat}{1}
        \varphi^{k+1}(t) &= 
            \sum_{i=0}^{k}
                \combination{k}{i}
                \bigg\{
                \frac{\partial^{k+1} f}{\partial x^{k+1-i} \partial y^{i}} (x(t), y(t))
                \Delta x^{k+1-i}
                \Delta y^{i} \nonumber \\
        &\quad +
                \frac{\partial^{k+1} f}{\partial x^{k-i} \partial y^{i+1}} (x(t), y(t))
                \Delta x^{k-i}
                \Delta y^{i+1}
                \bigg\} \\
        &= 
            \sum_{i=0}^{k}
                \combination{k}{i}
                \frac{\partial^{k+1} f}{\partial x^{k+1-i} \partial y^{i}} (x(t), y(t))
                \Delta x^{k+1-i}
                \Delta y^{i} \nonumber \\
        &\quad +
            \sum_{i=0}^{k}
                \combination{k}{i}
                \frac{\partial^{k+1} f}{\partial x^{k-i} \partial y^{i+1}} (x(t), y(t))
                \Delta x^{k-i}
                \Delta y^{i+1} \\
        % 一段ずらす
        &= 
            \sum_{i=0}^{k}
                \combination{k}{i}
                \frac{\partial^{k+1} f}{\partial x^{k+1-i} \partial y^{i}} (x(t), y(t))
                \Delta x^{k+1-i}
                \Delta y^{i} \nonumber \\
        &\quad +
            \sum_{i=1}^{k+1}
                \combination{k}{i-1}
                \frac{\partial^{k+1} f}{\partial x^{k+1-i} \partial y^{i}} (x(t), y(t))
                \Delta x^{k+1-i}
                \Delta y^{i} \\
        &= 
            \frac{\partial^{k+1} f}{\partial x^{k+1}} (x(t), y(t))
            \Delta x^{k+1}
            +
            \frac{\partial^{k+1} f}{\partial y^{k+1}} (x(t), y(t))
            \Delta y^{k+1} \nonumber \\
        &\quad +
            \sum_{i=1}^{k}
                \combination{k+1}{i}
                \frac{\partial^{k+1} f}{\partial x^{k+1-i} \partial y^{i}} (x(t), y(t))
                \Delta x^{k+1-i}
                \Delta y^{i} \\
        &=
            \sum_{i=0}^{k+1}
                \combination{k+1}{i}
                \frac{\partial^{k+1} f}{\partial x^{k+1-i} \partial y^{i}} (x(t), y(t))
                \Delta x^{k+1-i}
                \Delta y^{i}
    \end{alignat}
    したがって$k+1$に対しても\cref{lemma:1}が成立する。
    以上より、自然数$k=1,2,\dots,n+1$に対して\cref{lemma:1}が成立する。
\end{proof}

% ==========
% Section 9
% ==========
\section{$C^2$級の2変数関数の極大・極小}

\subsection{極値を取るための必要条件}
次の命題は「すぐにわかる」ものです。
\begin{screen}
    \begin{lemma}
        点$(x_0, y_0)$で$f$が極大、または極小になるならば
        \begin{equation}
            \frac{\partial f}{\partial x}(x_0, y_0) = 0
            \quad \mbox{かつ} \quad
            \frac{\partial f}{\partial y}(x_0, y_0) = 0
        \end{equation}
        である。ただし$x_0, y_0, f$は講義資料で定義されたものである。
    \end{lemma}
\end{screen}

\begin{proof}
    点$(x_0, y_0)$が$f$の極大点である場合を示す。極小点の場合も同様である。
    一変数関数$g, h$を$g(x) = f(x, y_0),\,h(y) = f(x_0, y)$と定めると、
    $g, h$はそれぞれ点$x_0, y_0$で微分可能であって局所的に最大となる。
    したがって、共通資料第4章命題2より
    点$x_0, y_0$はそれぞれ$g, h$の停留点である。よって
    \begin{equation}
        \frac{\partial f}{\partial x}(x_0, y_0) = g'(x_0) = 0
        ,\quad
        \frac{\partial f}{\partial y}(x_0, y_0) = h'(y_0) = 0
    \end{equation}
    が従う。
\end{proof}

\subsection{$A, B, C$ と $a, b, c$ の関係性}
次の命題は、$A, B, C$に関する大小関係がまわりの$a, b, c$に関しても成立するというものです。
\begin{screen}
    \begin{lemma}
        $(x_0, y_0)$に十分近い任意の点$(x, y)$に対して
        \begin{equation}
            A > 0
            \,\mbox{かつ}\,
            AC - B^2 > 0
            \quad \Longrightarrow \quad
            \mbox{常に}\,
            a > 0
            \,\mbox{かつ}\,
            ac - b^2 > 0
        \end{equation}
        が成立する。ただし$x, y, x_0, y_0, A, B, C, a, b, c$は講義資料で定義されたものである。
    \end{lemma}
\end{screen}

\begin{proof}
    $(x, y) \rightarrow (x_0, y_0)$で
    \begin{alignat}{1}
        a &\rightarrow A > 0 \\
        ac - b^2 &\rightarrow AC - B^2 > 0
    \end{alignat}
    なので、ある$\delta > 0$が存在して
    \begin{equation}
        0 < \| (x, y) - (x_0, y_0) \| < \delta
        \quad \Longrightarrow \quad
        a > 0
        \,\mbox{かつ}\,
        ac - b^2 > 0
    \end{equation}
    が成立する。
\end{proof}

\subsection{「$|t|$: 十分小」の意味}
ところで「$AC-B^2<0$の場合」というスライドでは「$|t|$: 十分小」という制約が登場しますが、
これは何のためにあるのか気になりませんか?
O先生に尋ねたところ次のような回答でした。証明はとくにありません。
\begin{screen}
    \begin{lemma}
        「$|t|$: 十分小」という制約は、$f$が$C^2$級である領域の上を点$(x_0 + t, y_0)$が動くように課されている。
    \end{lemma}
\end{screen}

\subsection{「iii) $A=0$のとき」の省略された計算)}
以下の計算が省略されています。
\begin{screen}
    \begin{lemma}
        講義資料のように
        \begin{equation}
            \varphi(t) = f(x_0 + p_1 t, y_0 + t),
            \quad
            \psi(t)    = f(x_0 + p_2 t, y_0 + t)
        \end{equation}
        とおくと次が成り立つ。
        \begin{equation}
            \varphi'(0) = \psi'(0) = 0,
            \quad
            \varphi''(0) = 2p_1 B + C,
            \quad
            \psi''(0)    = 2p_2 B + C
        \end{equation}
        ただし$x_0, y_0, p_1, p_2, B, C$は講義資料で定義されたものである。
    \end{lemma}
\end{screen}

\begin{proof}
    $\varphi$ についてのみ示す。$\psi$ の場合も同様である。
\end{proof}

\end{document}
