\documentclass[./index]{subfiles}
\begin{document}

\section{級数と冪級数}

% ==========
% 5.1
% ==========
\subsection{$\mathbb{R}$における級数の収束と発散}
\subsubsection{高校数学でやったこと④}
\begin{screen}
    \begin{proposition}
        \begin{equation}
            \sum_{n = 1}^{\infty} a_n \mbox{ が収束する}
            \quad \Longrightarrow \quad
            \lim_{n \rightarrow \infty} a_n = 0
        \end{equation}
    \end{proposition}
\end{screen}

\begin{proof}
    正数$\epsilon$を任意にとる.
    $\sum_{n = 1}^{\infty} a_n$が収束するから,
    ある自然数$N$が存在して,
    $m > N$なる任意の自然数$m$に対して
    \begin{equation}
        \left| \sum_{n = 1}^{m} a_n - S \right| < \frac{\epsilon}{2}
    \end{equation}
    が成立する.ただし,$S = \sum_{n = 1}^{\infty} a_n$である.
    よって,ある自然数$N$が存在して,
    $m > N$なる任意の自然数$m$に対して
    \begin{equation}
        \begin{cases}
            \displaystyle S - \frac{\epsilon}{2} < \sum_{n = 1}^{m} a_n < S + \frac{\epsilon}{2} \\
            \displaystyle S - \frac{\epsilon}{2} < \sum_{n = 1}^{m - 1} a_n < S + \frac{\epsilon}{2}
        \end{cases}
    \end{equation}
    が成立し,したがって
    \begin{equation}
        |a_m| = \left| \sum_{n = 1}^{m} a_n - \sum_{n = 1}^{m - 1} a_n \right| < \epsilon
    \end{equation}
    が成立する.
    よって,
    \begin{equation}
        \lim_{n \rightarrow \infty} a_n = 0
    \end{equation}
    が成立する.
\end{proof}

\end{document}