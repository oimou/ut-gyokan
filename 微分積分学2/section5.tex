\documentclass[./index]{subfiles}
\begin{document}

\section{級数と冪級数}

% ==========
% 5.1
% ==========
\subsection{$\mathbb{R}$における級数の収束と発散}
\subsubsection{高校数学でやったこと④}
\begin{screen}
    \begin{proposition}
        \begin{equation}
            \sum_{n = 1}^{\infty} a_n \mbox{ が収束する}
            \quad \Longrightarrow \quad
            \lim_{n \rightarrow \infty} a_n = 0
        \end{equation}
    \end{proposition}
\end{screen}

\begin{proof}
    級数$\sum_{n=1}^\infty a_n$の値を$\alpha$とおき,第$n$部分和を$S_n$とおくと,
    \begin{equation}
        \lim_{n \rightarrow \infty} S_n = \lim_{n \rightarrow \infty} S_{n-1} = \alpha
    \end{equation}
    であるから
    \begin{equation}
        \lim_{n \rightarrow \infty} a_n
        = \lim_{n \rightarrow \infty} (S_n - S_{n-1})
        = \alpha - \alpha
        = 0
    \end{equation}
    である.
\end{proof}

% ==========
% 5.6
% ==========
\setcounter{subsection}{5}
\subsection{収束冪級数の微分と積分}
\subsubsection{$r \le r'$の証明}
\begin{screen}
    \begin{proposition}
        $c, r, r'$を実数とし,任意の実数$x$に対して
        \begin{equation}
            \label{eq:5:6:1}
            |x - c| < r \quad \mbox{ならば} \quad |x - c| \le r'
        \end{equation}
        が成り立つとする.このとき$r \le r'$である.
    \end{proposition}
\end{screen}

講義資料では「ここで,$x$は$|x - c| < r$となる任意の実数であったから,$r \le r'$と分かる.」
と述べられている部分です.より厳密には背理法で示すことができます.

\begin{proof}
    $r > r'$と仮定すると,稠密性により$r' < |x - c| < r$なる実数$x$が存在するが,
    これは\cref{eq:5:6:1}に矛盾する.
\end{proof}

\end{document}